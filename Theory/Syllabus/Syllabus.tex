\documentclass[a4paper,11pt]{article}

\usepackage[T1]{fontenc}
\usepackage[utf8]{inputenc}
\usepackage{graphicx}
\usepackage{subfigure}
\usepackage{xcolor}

\renewcommand\familydefault{\sfdefault}
\usepackage{tgheros}
%\usepackage[defaultmono]{droidmono}

\usepackage{amsmath,amssymb,amsthm,textcomp}
\usepackage{enumerate}
\usepackage{multicol}
\usepackage{tikz}

\usepackage{enumitem}
\newlist{legal}{enumerate}{10}
\setlist[legal]{label*=\arabic*.}

\usepackage{geometry}
\geometry{total={210mm,297mm},
left=25mm,right=25mm,%
bindingoffset=0mm, top=20mm,bottom=20mm}


\linespread{1.3}

\newcommand{\linia}{\rule{\linewidth}{0.5pt}}

% custom theorems if needed
\newtheoremstyle{mytheor}
    {1ex}{1ex}{\normalfont}{0pt}{\scshape}{.}{1ex}
    {{\thmname{#1 }}{\thmnumber{#2}}{\thmnote{ (#3)}}}

\theoremstyle{mytheor}
\newtheorem{defi}{Definition}

% my own titles
\makeatletter
\renewcommand{\maketitle}{
\begin{center}
\vspace{2ex}
{\huge \textsc{\@title}}
\vspace{1ex}
\\
\linia\\
\@author \hfill \@date
\vspace{4ex}
\end{center}
}
\makeatother
%%%

% custom footers and headers
\usepackage{fancyhdr}
\pagestyle{fancy}
\lhead{}
\chead{}
\rhead{}
\lfoot{Syllabus-2020-1}
\cfoot{}
\rfoot{Página \thepage}
\renewcommand{\headrulewidth}{0pt}
\renewcommand{\footrulewidth}{0pt}
%

% code listing settings
\usepackage{listings}
\lstset{
    language=Python,
    basicstyle=\ttfamily\small,
    aboveskip={1.0\baselineskip},
    belowskip={1.0\baselineskip},
    columns=fixed,
    extendedchars=true,
    breaklines=true,
    tabsize=4,
    prebreak=\raisebox{0ex}[0ex][0ex]{\ensuremath{\hookleftarrow}},
    frame=lines,
    showtabs=false,
    showspaces=false,
    showstringspaces=false,
    keywordstyle=\color[rgb]{0.627,0.126,0.941},
    commentstyle=\color[rgb]{0.133,0.545,0.133},
    stringstyle=\color[rgb]{01,0,0},
    numbers=left,
    numberstyle=\small,
    stepnumber=1,
    numbersep=10pt,
    captionpos=t,
    escapeinside={\%*}{*)}
}

%%%----------%%%----------%%%----------%%%----------%%%

\begin{document}

\title{Computación Gráfica Avanzada \\ Syllabus-2020-1}

\author{M.C. Reynaldo Martell Avila, Grupo \textnumero{1} , Martes y Jueves 13:00 - 14:30}

\date{\today}

\maketitle

\section*{Temario}

Introducción a la materia.\textbf{(6 de Agosto)}
\begin{legal}
	\item Introducción a OpenGL. \textbf{(6 de Agosto)}
	\begin{legal}
		\item Introducción e Historia de OpenGL 3.3. \textbf{(8 de Agosto)}
		\item Integración de librerías para el desarrollo con OpenGL 3.3.
		\begin{legal}
			\item Integración de GLEW y GLFW para la creación de una ventana con
OpenGL 3.3.
			\item Manejo de eventos con GLFW.
		\end{legal}
	\end{legal}
	\item OpenGL 3.3 \textbf{(13, 15 y 20 Agosto)}
	\begin{legal}
		\item Pipeline de renderizado.
		\item Primer programa con OpenGL 3.3
		\item VAO y VBO
		\item Índices.
		\item Shaders.
		\item Transformaciones.
	\end{legal}
	\item Texturizado Avanzado Parte 1. \textbf{(22 y 27 Agosto)}
	\begin{legal}
		\item Texturizado shaders.
		\item Multi-texturas.
	\end{legal}
	\item Animación Avanzada y ambientes interactivos Parte 1. \textbf{(3 y 5 de
Septiembre)}
	\begin{legal}
		\item Proyección Ortogonal, perspectiva, puerto de Vista.
		\item Cámaras.
		\item Carga de modelos con Assimp.
	\end{legal}
	\item Iluminación Avanzada Parte 1. \textbf{(10 febrero, 12 y 17 Septiembre)}
	\begin{legal}
		\item Iluminación Básica.
		\item Materiales.
		\item Mapas de iluminación.
		\item Tipos de luces.
	\end{legal}
		\textbf{Examen 19 Septiembre}
	\item Texturizado Avanzado Parte 2. \textbf{(24 y 26 Septiembre)}
	\begin{legal}
		\item Texturas 3D.
		\item Cube Mapping. Reflexión y Refracción
		\item Bump mapping.
	\end{legal}
	\item Colisiones. \textbf{(1, 3 y 8 de Octubre)}
	\begin{legal}
		\item Colisión linea vs linea.
		\item Colisión caja vs punto.
		\item Colisión caja vs caja.
		\item Colisión esfera vs esfera.
		\item Colisión esfera vs plano.
		\item Colisión esfera vs caja.
		\item Colisión rayo vs esfera.
		\item Colisión caja vs rayo.
	\end{legal}
	\item Animación Avanzada y ambientes interactivos Parte 2. \textbf{(10, 15, 17 y 22 de Octubre)}
	\begin{legal}
		\item Picking.
		\item Audio con OpenAl.
		\item Partículas.
		\item 
	\end{legal}
	\item OpenGL avanzado. \textbf{(24, 39, 31 de Octubre y 5 de Noviembre)}
	\begin{legal}
		\item Depth buffer
		\item Stencil buffer.
		\item Blending.
		\item Anti aliasing.
	\end{legal}
	\item Shadow mapping.
	\item Iluminación Avanzada Parte 2. \textbf{(7, 12 , 14 y 19 de Noviembre)}
	\begin{legal}
		\item Bloom
		\item Radiosidad
		\item Bling Phong
	\end{legal}
	\textbf{Segundo examen parcial. 21 de Noviembre}
	\\ \textbf{Entrega del proyecto. 28 de Noviembre }
	\\ \textbf{Examen Final. 3 de Diciembre}
	
\end{legal}

\section*{EVALUACIÓN}

\begin{itemize}
	\item Exámenes.	40 \%
	\begin{itemize}
		\item Parcial 1
		\item Parcial 2
	\end{itemize}
	\item Proyecto 40 \%
	\item Tareas e investigaciones 20 \%
\end{itemize}

\section*{EVALUACIÓN}

\begin{itemize}
	\item ANGEL, Edward, Interactive Computer Graphics: A Top-Down Approach
with OpenGL 4, 6ta edition, Portland
Addison-Wesley. 2011.
	\item Alab B. Craig, William R. Sherman, Jeffrey D. Will, Developing Virtual
Reality Applications, Elsevier, 2009
	\item Mario A. Gutiérrez A. Frédéric Vexo, Daniel Thalmann, Stepping into Virtual
Reality, Springer, 2008.
	\item Mark Segal, Kurt Akeley, The OpenGLR Graphics System Version 3.3 (Core
Pro le) The Khronos Group, 2011.
	\item Wilbert O. Galitz, The Essential Guide to User Interface Design, Wiley
Computer Publishing, Second Edition, 2002.
	\item Dave Shreiner, Graham Seliers, John Kessenich, Bill Licea-Kane,
Programming Guide The oficial Guide to Learning
OpenGL Version
4.3, The Khronos Group, Eighth Edition.
	\item David Wolff, OpenGL 4.0 Shading Language Cookbook, Packt publishing,
2011.
	\item Christer Ericson, Real-Time Collision Detection, Sony Computer
Entertainment America.
\end{itemize}

\end{document}
