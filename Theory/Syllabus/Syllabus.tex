\documentclass[a4paper,11pt]{article}

\usepackage[T1]{fontenc}
\usepackage[utf8]{inputenc}
\usepackage{graphicx}
\usepackage{subfigure}
\usepackage{xcolor}

\renewcommand\familydefault{\sfdefault}
\usepackage{tgheros}
%\usepackage[defaultmono]{droidmono}

\usepackage{amsmath,amssymb,amsthm,textcomp}
\usepackage{enumerate}
\usepackage{multicol}
\usepackage{tikz}

\usepackage{enumitem}
\newlist{legal}{enumerate}{10}
\setlist[legal]{label*=\arabic*.}

\usepackage{geometry}
\geometry{total={210mm,297mm},
left=25mm,right=25mm,%
bindingoffset=0mm, top=20mm,bottom=20mm}


\linespread{1.3}

\newcommand{\linia}{\rule{\linewidth}{0.5pt}}

% custom theorems if needed
\newtheoremstyle{mytheor}
    {1ex}{1ex}{\normalfont}{0pt}{\scshape}{.}{1ex}
    {{\thmname{#1 }}{\thmnumber{#2}}{\thmnote{ (#3)}}}

\theoremstyle{mytheor}
\newtheorem{defi}{Definition}

% my own titles
\makeatletter
\renewcommand{\maketitle}{
\begin{center}
\vspace{2ex}
{\huge \textsc{\@title}}
\vspace{1ex}
\\
\linia\\
\@author \hfill \@date
\vspace{4ex}
\end{center}
}
\makeatother
%%%

% custom footers and headers
\usepackage{fancyhdr}
\pagestyle{fancy}
\lhead{}
\chead{}
\rhead{}
\lfoot{Syllabus-2020-1}
\cfoot{}
\rfoot{Página \thepage}
\renewcommand{\headrulewidth}{0pt}
\renewcommand{\footrulewidth}{0pt}
%

% code listing settings
\usepackage{listings}
\lstset{
    language=Python,
    basicstyle=\ttfamily\small,
    aboveskip={1.0\baselineskip},
    belowskip={1.0\baselineskip},
    columns=fixed,
    extendedchars=true,
    breaklines=true,
    tabsize=4,
    prebreak=\raisebox{0ex}[0ex][0ex]{\ensuremath{\hookleftarrow}},
    frame=lines,
    showtabs=false,
    showspaces=false,
    showstringspaces=false,
    keywordstyle=\color[rgb]{0.627,0.126,0.941},
    commentstyle=\color[rgb]{0.133,0.545,0.133},
    stringstyle=\color[rgb]{01,0,0},
    numbers=left,
    numberstyle=\small,
    stepnumber=1,
    numbersep=10pt,
    captionpos=t,
    escapeinside={\%*}{*)}
}

%%%----------%%%----------%%%----------%%%----------%%%

\begin{document}

\title{Computación Gráfica Avanzada \\ Syllabus-2020-2}

\author{M.C. Reynaldo Martell Avila, Grupo \textnumero{1} , Martes y Jueves 13:00 - 15:00}

\date{\today}

\maketitle

\section*{Temario}

Introducción a la materia.\textbf{(28 de Enero)}
\begin{legal}
	\item Introducción a OpenGL. \textbf{(28 - 30 Enero)}
	\begin{legal}
		\item Introducción e Historia de OpenGL 3.3.
		\item Pipeline de renderizado.
		\item Shaders.
		\item Transformaciones.
	\end{legal}
	\item OpenGL 3.3 \textbf{(6 Febrero y 13 Febrero)}
	\begin{legal}
		\item Quaterniones.
		\item Coordenadas baricentricas.
		\item Mapas de alturas.
		\item Interpolación de alturas en coordenadas baricentricas.
		\item Interpolación de normales en coordenadas baricentricas.
	\end{legal}
	\item Texturizado e iluminación avanzado. \textbf{(20 Febrero y 27 Febrero)}
	\begin{legal}
		\item Texturizado 3D.
		\item Multiples texturas.
		\item Iluminación: tipos de luces.
		\item Cámaras.
	\end{legal}
	\item Colisiones. \textbf{(5, 12, 17 y 19 Marzo)}
	\begin{legal}
		\item Colisión Linea vs Linea 2D.
		\item Colisión Linea vs Esfera.
		\item Colisión Rayo vs Esfera.
		\item Colisión Rayo vs AABB.
		\item Colisión Rayo vs Triánguo.
		\item Colisión Esfera vs Plano.
		\item Colisión Esfera vs Esfera.
		\item Colisión AABB vs AABB.
		\item Colisión AABB vs Esfera.
		\item Colisión OBB vs Sphere.
		\item Colisión OBB vs OBB.
		\item Prueba Sweep sphere vs Sphere.
	\end{legal}
	\textbf{Primer examen parcial. 26 de Marzo}
	\item OpenGL avanzado. \textbf{(31 Marzo, 14 y 21 Abril)}
	\begin{legal}
		\item Framebuffer.
		\item Buffer de profundidad.
		\item Blending.
		\item Niebla.
		\item Particulas.
		\item Particulas retroalimentadas.
		\item Shadow mapping.
	\end{legal}	
	\item Curvas paramétricas. \textbf{(25 Abril y 5 Mayo)}
	\begin{legal}
		\item Curvas de bézier.
		\item Splines.
	\end{legal}
	\textbf{Segundo examen parcial. 19 de Mayo}
	\\ \textbf{Entrega del proyecto. 26 de Mayo }
	\\ \textbf{Primer examen Final. 28 de Mayo}
	
\end{legal}

\section*{Prácticas}
\begin{legal}
	\item Practica 1: Integración de librerías para el desarrollo con OpenGL 3.3. \textbf{(4 Febrero)}
	\item Practica 2: Animaciones por esqueletos. \textbf{(11 Febrero)}
	\item Practica 3: Terrenos. \textbf{(18 Febrero)}
	\item Practica 4: Terrenos y multiples texturas. \textbf{(25 Febrero)}
	\item Practica 5: Multiples luces. \textbf{(3 Marzo)}
	\item Practica 6: Cámara en tercera persona. \textbf{(10 Febrero)}
	\item Practica 7: Colisiones. \textbf{(24 Marzo)}
	\item Practica 8: Buffer de profundida. \textbf{(2 Abril)}
	\item Practica 9: Blending \textbf{(16 Abril)}.
	\item Practica 10: Fog \textbf{(23 Abril)}.
	\item Practica 11: Simulación de fuente de agua. \textbf{(30 Abril)}.
	\item Practica 12: Simulación de fuego. \textbf{(7 Mayo)}.
	\item Practica 13: Sombras. \textbf{(12 Mayo)}.
	\item Practica 14: OpenAL y sombras. \textbf{(14 Mayo)}.
\end{legal}


\section*{EVALUACIÓN}

\begin{itemize}
	\item Exámenes.	30 \%
	\begin{itemize}
		\item Parcial 1
		\item Parcial 2
	\end{itemize}
	\item Proyecto 35 \%
	\item Prácticas 25 \%
	\item Tareas e investigaciones 10 \%
\end{itemize}

\section*{Bibliografía}

\begin{itemize}
	\item ANGEL, Edward, Interactive Computer Graphics: A Top-Down Approach
with OpenGL 4, 6ta edition, Portland
Addison-Wesley. 2011.
	\item Alab B. Craig, William R. Sherman, Jeffrey D. Will, Developing Virtual
Reality Applications, Elsevier, 2009
	\item Mario A. Gutiérrez A. Frédéric Vexo, Daniel Thalmann, Stepping into Virtual
Reality, Springer, 2008.
	\item Mark Segal, Kurt Akeley, The OpenGLR Graphics System Version 3.3 (Core
Pro le) The Khronos Group, 2011.
	\item Wilbert O. Galitz, The Essential Guide to User Interface Design, Wiley
Computer Publishing, Second Edition, 2002.
	\item Dave Shreiner, Graham Seliers, John Kessenich, Bill Licea-Kane,
Programming Guide The oficial Guide to Learning
OpenGL Version
4.3, The Khronos Group, Eighth Edition.
	\item David Wolff, OpenGL 4.0 Shading Language Cookbook, Packt publishing,
2011.
	\item Christer Ericson, Real-Time Collision Detection, Sony Computer
Entertainment America.
\end{itemize}

\end{document}
